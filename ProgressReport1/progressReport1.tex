\documentclass[pdftex,12pt,letter]{article}

\usepackage{enumerate}
\makeatletter
  \renewcommand\@seccntformat[1]{\csname the#1\endcsname.\quad}
\makeatother
\newcommand{\HRule}{\rule{\linewidth}{0.5mm}}
\begin{document}

\begin{titlepage}
\begin{flushright}
\HRule \\
{\huge \bfseries Case Scheduler\\[4cm]}
{\large Prepared by\\Jason Kuster, Stuart Long, and Nathan McKinley\\[1cm]
October 1, 2012}
\end{flushright}
\end{titlepage}
%\tableofcontents{}
%\newpage
\section*{Introduction Overview}
Currently, there only exist functional but not particularly elegant ways for students to visualize their class schedules. For our project, we want to make a better Case Scheduler, one which works fast and is easy for students to use.\\

\noindent When making a class schedule, it is very helpful to be able to see what your schedule looks like. The two systems currently available, SIS and scheduler.case.edu do an unsatisfactory job of displaying the schedule. The problem with SIS is that it's exceedingly slow and inefficient for course planning, and is much more suited to only doing course enrollment. The problem with scheduler.case.edu is that not only is it slow to open and do anything, but it is impossible to print without using the Print Screen function on your computer, exporting it to an image, and printing that image. In addition, there are many expansions which could be made to functionality, such as easy sharing of schedules and planning your schedule with friends.\\

\noindent We will be using two technologies to implement this scheduler. First, we will be using Python and the Django framework for programming our web application. Second, we will be using MySQL for our database.
\section*{Application Requirements Specifications}
\begin{enumerate}[1.]
\item Maintains a per user list of courses
\item Searches for courses
\item Add course to users' list
\item Remove course from users' list
\item Display course information on a calendar-formatted schedule, including course name, times, instructor, and location.
\item Allows course schedule to be easily printable
\item Maintains users' lists across SIS updates
\item Displays more detailed course information when requested
\item Low response times under reasonable user load (less than 1 second)
\item Allows users to have multiple course lists corresponding to different semesters
\item Supports user specified custom events, which will be treated like new courses
\end{enumerate}
\section*{Database Requirements Specifications}
\subsection*{Objects and Relationships}
\begin{enumerate}[1.]
\item Course Offering: The database will have an instance of course offering for each course offered by CWRU. We considered having a overreaching "Course" object where Course Offering is-a Course. However since the purpose of our application is mainly as a scheduler we are more concerned with attribute specific to each offering, namely when the course is held, we decided to just have the course offering. A course offering will have at least the following attributes: offering id, term, subject, catalog number, title, label, component, status, dates, units (max), days/times, room, total capacity, current enrollment, prerequisites, and description.\\\\ In explanation, the "label" attribute corresponds to a short combination of the of the subject, catalog number, and title. The catalog number corresponds to the course number within that courses department (e.g. 314 for EECS 341). The "component" attribute represents the type of the course, whether it's a lecture, recitation, or similar. The "status" attribute represents whether the course is  open, closed, or other. "Current Enrollment" signifies how many other students are registered for the course offering. "Description" is a paragraph explanation of the course provided by the department or professor.
\item Users
\item Instructors
\item Teaches Relationship
\item Enrolled-in Relationship
\end{enumerate}
\subsection*{Queries and Transactions}
\begin{enumerate}[1.]
\item Get\_List Query - Very frequent, once ppv
\item Lookup\_Course - Even more frequent, multiples times ppv
\item Add\_Course - Less frequent, less than once ppv
\item Remove\_Course - Infrequent, much less than once ppv
\item Lookup\_Instructor - Infrequent, less than 1 in 10 pv's
\end{enumerate}
\subsection*{Actions and Events}
\begin{enumerate}[1.]
\item Select Semester - displays a different schedule depending on the selected semester
\item User login - displays the user's last used schedule
\end{enumerate}
\subsection*{Integrity Constraints}
\begin{enumerate}
\item Users must have at least one course, or the user is deleted
\item Lists of courses cannot be more than 4 years old
\item Users cannot be inactive for more than 4 years
\item Courses with no users can be removed after the semester during which they were active
\end{enumerate}

\end{document}